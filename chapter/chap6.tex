% !TeX root = ../main.tex
% !TEX root = ../main.tex
% -*- root: ../main.tex -*-
% -*- program: pdflatex -*-
\chapter{总结}
本文主要介绍了端盖飞行时间探测器MRPC刻度方法的研究。MRPC探测器具有时间分辨小,探测效率高,造价低廉等优点。2015年夏季~BESIII~的端盖飞行时间探测器完成了升级改造,由原来的闪烁体换成了~MRPC~,并在2015-2016年完成第一轮取数。分析并研究新的端盖~MRPC~飞行时间探测器的刻度方法,得到好的时间分辨,对于之后的粒子鉴别和物理分析都很有意义。刻度选用的是纯度高的Bhabha事例。主要研究的内容包括信号在读数条的传播时间,已经过阈时间的修正问题。对于在读数条的传播时间问题,相对简单,采用多项式拟合即可完成。而对于过阈时间的修正问题,相对复杂,这也是本文研究的重点和难点。文中从样条插值方法和构造公式方法两种方法介绍了刻度方法。

STAR实验采用的就是样条插值方法。在本文的第三章讨论了插值方法在~BESIII~实验~MRPC~离线数据刻度中的应用。经过对过阈时间和击中位置修正先后顺序的研究。发现在~BESIII~实验上采用先对击中位置进行修正,然后对过阈时间采用样条插值得到的时间分辨好。

之后的第四章讨论了构造公式在~MPRC~离线刻度中的应用。对于击中位置的修正采用的是多项式拟合。在此修正基础上,分析时间与过阈时间的分布关系,拟构造了击中简单的公式。然后进行拟合并比较。

对于双端,采用和单端研究类似的方法。由于径迹外推在边界上存在误差。对于双端的击中位置的修正,采用了两端原始时间相减的平均值替代外推得到的击中位置进行修正。

最后分析刻度公式的适用性问题,并研究了过阈时间和击中位置关联项在刻度公式中的影响。













