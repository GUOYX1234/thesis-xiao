% !TeX root = ../main.tex
% !TEX root = ../main.tex
% -*- root: ../main.tex -*-
% -*- program: pdflatex -*-
\chapter{双端修正}
上两章主要从插值方法和构造公式入手介绍了对TOF的MRPC的离线数据的刻度方法,研究的主要是单端的刻度。这章将介绍双端刻度方法。

\begin{figure}[!h]
\begin{minipage}[!h]{0.5\linewidth}
%\centering
\includegraphics[width=0.9\textwidth]{chap4/combined-tVSz.eps}
\subcaption{双端时间对~Z~的分布}
\label{fig:combined-tVSz}
\end{minipage}%
\hfill
\begin{minipage}[!h]{0.5\linewidth}
%\centering
\includegraphics[width=0.9\textwidth]{chap4/combined-tVSq.eps}
\subcaption{时间对~TOT~的分布}
\label{fig:combined-tVSq}
\end{minipage}
\caption{双端时间对~Z~和~TOT~的分布}
\end{figure}

图~\ref{fig:combined-tVSz}~和图~\ref{fig:combined-tVSq}~是双端的时间对~Z~和~TOT~的分布。所谓双端指的是两端时间的平均值,~TOT~采用的也是两端~TOT~的平均值。从图中可以看出,双端的时间对~Z~的依赖很小,主要就是时间对~TOT~的依赖关系。而这时的时间对~TOT~的依赖关系和单端修正~Z~后时间对~TOT~的分布很类似。因此对于双端修正,直接对~TOT~修正,采用和单端修正~Z~后处理时间与~TOT~的关系相同的办法。

\section{双端插值方法}

\begin{figure}[!h]
\begin{minipage}[!h]{0.5\linewidth}
%\centering
\includegraphics[width=0.9\textwidth]{chap4/combined-splines.eps}
\subcaption{双端插值}
\label{fig:combined-splines}
\end{minipage}%
\hfill
\begin{minipage}[!h]{0.5\linewidth}
%\centering
\includegraphics[width=0.9\textwidth]{chap4/combined-splines-line.eps}
\subcaption{双端插值线}
\label{fig:combined-splines-line}
\end{minipage}
\caption{双端插值方法}
\end{figure}

图~\ref{fig:combined-splines}~和图~\ref{fig:combined-splines-line}~表示的是双端的插值方法。图~\ref{fig:combined-splines}~采用的是和单端处理时间和~TOT~的一样的方法:将时间和~TOT~按~TOT~大小等事例数分成100份,对每个~bin~拟合得到中心值。图~\ref{fig:combined-splines-line}~表示的是对此100个点采用三阶样条插值拟合得到的曲线。

\section{双端构造公式}

图~\ref{fig:single-formula}~是上述几种公式拟合的结果。
就这一条的拟合结果来看,公式~\ref{eq:5}~,也就是${p_{0}+p_{1}/\sqrt{q}+p_{2}/q}$的拟合最好。

\begin{figure}[!h]
\begin{minipage}{0.5\linewidth}
  \centerline{\includegraphics[width=0.9\textwidth]{chap4/double-half-order.eps}}
  \centerline{(a) 公式~\ref{eq:1}~的拟合}
  \centerline{\label{fig:double-half-order}}
\end{minipage}
\hfill
\begin{minipage}{0.5\linewidth}
  \centerline{\includegraphics[width=0.9\textwidth]{chap4/double-1-order.eps}}
  \centerline{(b) 公式~\ref{eq:2}~的拟合}
  \centerline{\label{fig:double-1-order}}
\end{minipage}
\vfill
\begin{minipage}{0.5\linewidth}
  \centerline{\includegraphics[width=0.9\textwidth]{chap4/double-2-order.eps}}
  \centerline{(c) 公式~\ref{eq:3}~的拟合}
  \centerline{\label{fig:double-2-order}}
\end{minipage}
\hfill
\begin{minipage}{0.5\linewidth}
  \centerline{\includegraphics[width=0.9\textwidth]{chap4/double-3-order.eps}}
  \centerline{(d) 公式~\ref{eq:4}~的拟合}
  \centerline{\label{fig:double-3-order}}
\end{minipage}
\vfill
\begin{minipage}{0.5\linewidth}
  \centerline{\includegraphics[width=0.9\textwidth]{chap4/double-half-1-order.eps}}
  \centerline{(e) 公式~\ref{eq:5}~的拟合}
  \centerline{\label{fig:double-half-1-order}}
\end{minipage}
\hfill
\begin{minipage}{0.5\linewidth}
  \centerline{\includegraphics[width=0.9\textwidth]{chap4/double-pol3-order.eps}}
  \centerline{(f) 公式~\ref{eq:6}~的拟合}
  \centerline{\label{fig:double-pol3-order}}
\end{minipage}
\caption{几种公式对~TOT~的拟合}
\label{fig:single-formula}
\end{figure}

表~\ref{tbl:combined-resolution}~给出了这几种公式修正完成后得到的时间分辨。可以看出公式${p_{0}+p_{1}/\sqrt{q}+p_{2}/q}$的结果是最好的。

在此研究的基础上随机选了两个模块,然后比较几种公式修正得到的时间分辨的。见图~\ref{fig:double-module45}~和图~\ref{fig:double-module50}~。可以看出,除了使用多项式外,公式${p_{0}+p_{1}/\sqrt{q}+p_{2}/q}$得到的时间分辨基本是最好的。而不选用多项式的原因是因为,多项式拟合在~TOT~比较大的一端拟合的不好,并未真正的反映出时间随~TOT~在大~TOT~下趋于平滑的这种趋势。

\begin{table}[h]
    \centering
    \caption{\label{tbl:combined-resolution} 上述公式修正后最终的时间分辨}
  \footnotesize
    \begin{tabular}{lc}
        \hline
        公式& 时间分辨(ps) \\
        \hline
        ${p_{0}+p_{1}/\sqrt{q}}$ & 60.1 \\
        ${p_{0}+p_{1}/q}$ & 58.9 \\
        ${p_{0}+p_{1}/q^{2}}$ & 65.9 \\
        ${p_{0}+p_{1}/q^{3}}$ & 56.7 \\
        ${p_{0}+p_{1}/\sqrt{q}+p_{2}/q}$ & 56.4 \\
        ${p_{0}+p_{1}*q+p_{2}*q^{2}+p_{3}*q^3}$ & 56.6 \\
        \hline
    \end{tabular}
\end{table}

\begin{figure}[!h]
\begin{minipage}[!h]{0.5\linewidth}
%\centering
\includegraphics[width=0.9\textwidth]{chap4/double-module45.eps}
\subcaption{模块编号为45的几种公式时间分辨的比较}
\label{fig:double-module45}
\end{minipage}%
\hfill
\begin{minipage}[!h]{0.5\linewidth}
%\centering
\includegraphics[width=0.9\textwidth]{chap4/double-module50.eps}
\subcaption{模块编号为50的几种公式时间分辨的比较}
\label{fig:double-module50}
\end{minipage}
\caption{几种公式修正得到的时间分辨的比较}
\end{figure}

\begin{figure}[!h]
\begin{minipage}[!h]{0.5\linewidth}
%\centering
\includegraphics[width=0.9\textwidth]{chap4/Form-tVSz1.eps}
\subcaption{双端修正~TOT~后时间对~Z~的分布}
\label{fig:Form-tVSz1}
\end{minipage}%
\hfill
\begin{minipage}[!h]{0.5\linewidth}
%\centering
\includegraphics[width=0.9\textwidth]{chap4/Form-sigma1.eps}
\subcaption{双端修正~TOT~后的时间分辨}
\label{fig:Form-sigma1}
\end{minipage}
\caption{双端修正~TOT~后的分布}
\end{figure}

\section{双端对~Z~的修正}

图~\ref{fig:Form-tVSz1}~给出了双端时间修正过~TOT~后时间对~Z~的分布,可以看出还是有一定的依赖的。对此需要对击中位置~Z~进行修正。

%%这部分需要写完前面的径迹外推后再写。
击中位置~zrhit~来自于径迹外推。在读数条边界处会有偏差。为了避开使用~zrhit~,选择使用~(tleft-tright)/2~代替它。
tleft=(L+zrhit)/v+$C_{0}$;
tright=(L-zrhit)/v+$C_{1}$
两式相减除2得(tleft-tright)/2=zrhit/v+$C_{2}$
其中~L~为半个读数条长,为一常量;~v~为信号在读数条内的传播速度,为常量;$C_{0}$,$C_{1}$分别是两端的时延等信息,为常量。$C_{2}$=$C_{0}$-$C_{1}$;综上得出(tleft-tright)/2与~zrhit~成正相关。

图~\ref{fig:zVStsub}~是~tsub~和~zrhit~的分布,可以看出呈线性关系。
\begin{figure}[!h]
\centering
\includegraphics[width=0.6\textwidth]{chap4/zVStsub.eps}
\caption{~zrhit~与~(tleft-tright)/2~的分布}
\label{fig:zVStsub}
\end{figure}

基于此,可以利用(tleft-tright)/2信息修正zrhit。

\begin{figure}[!h]
\centering
\includegraphics[width=0.6\textwidth]{chap4/Form-Fit-Tsub.eps}
\caption{对~tsub~采用四阶多项式拟合}
\label{fig:Form-Fit-Tsub}
\end{figure}

图~\ref{fig:Form-Fit-Tsub}~是对~tsub~采用了一个四阶多项式的修正。采用四阶多项式的原因是,更低阶的多项式不能完全拟合时间和~tsub~的这种变化的关系。而更高阶多项式也不能得到比四阶多项式更好的拟合。

图~\ref{fig:Fit-Tsub}~是修正~tsub~前后,时间对~tsub~和~Z~的分布关系。可以看出,修正过~tsub~后,时间对~Z~的分布不再有明显的依赖。
\begin{figure}[!h]
\begin{minipage}{0.5\linewidth}
  \centerline{\includegraphics[width=0.9\textwidth]{chap4/before-offset-tsub.eps}}
  \centerline{(a) ~tsub~修正前时间对~tsub~的分布}
  \centerline{\label{fig:before-offset-tsub}}
\end{minipage}
\hfill
\begin{minipage}{0.5\linewidth}
  \centerline{\includegraphics[width=0.9\textwidth]{chap4/before-offset-z.eps}}
  \centerline{(b) ~tsub~修正前时间对~Z~的分布}
  \centerline{\label{fig:before-offset-z}}
\end{minipage}
\vfill
\begin{minipage}{0.5\linewidth}
  \centerline{\includegraphics[width=0.9\textwidth]{chap4/after-offset-tsub.eps}}
  \centerline{(c) ~tsub~修正后时间对~tsub~的分布}
  \centerline{\label{fig:after-offset-tsub}}
\end{minipage}
\hfill
\begin{minipage}{0.5\linewidth}
  \centerline{\includegraphics[width=0.9\textwidth]{chap4/after-offset-z.eps}}
  \centerline{(d) ~tsub~修正后时间对~Z~的分布}
  \centerline{\label{fig:after-offset-z}}
\end{minipage}
\caption{~tsub~修正前后时间对~tsub~和~Z~的分布}
\label{fig:Fit-Tsub}
\end{figure}

\begin{figure}[!h]
\begin{minipage}[!h]{0.5\linewidth}
%\centering
\includegraphics[width=0.9\textwidth]{chap4/Form-tVSz2.eps}
\subcaption{双端修正~tsub~后时间对~Z~的分布}
\label{fig:Form-tVSz2}
\end{minipage}%
\hfill
\begin{minipage}[!h]{0.5\linewidth}
%\centering
\includegraphics[width=0.9\textwidth]{chap4/Form-sigma2.eps}
\subcaption{双端修正~tsub~后的时间分辨}
\label{fig:Form-sigma2}
\end{minipage}
\caption{双端修正~tsub~后的分布}
\end{figure}

图~\ref{fig:Form-tVSz2}~是修正~tsub~后时间对~Z~的分布的散点图。可以看出这时候分布关系已经基本平滑。图~\ref{fig:Form-sigma2}~是修正~tub~后双端的时间分辨,为~53.5ps~。

\section{小结}

本章介绍了双端刻度方法的研究。在对~TOT~的修正上,采用的是和单端对此修正一样的方法。但对于~Z~的修正,由于边界效应,径迹外推得到的~zrhit~在读数条两端有偏差。基于此,采用了和~zrhit~类似的信息~(tleft-tright)/2~的信息对~Z~进行修正。最终得到双端的修正后的时间分辨等信息。











