% !TeX root = ../main.tex
% !TEX root = ../main.tex
% -*- root: ../main.tex -*-
% -*- program: pdflatex -*-
\chapter{总结}
本论文介绍了北京谱仪~MRPC~端盖飞行时间探测器刻度方法的研究。利用~2016~年~5~月~24~日到~5~月~30~日的北京正负电子对撞机对撞获取的~Bhabha~事例真实数据为刻度样本,分别利用样条插值方法和构造公式两种方法,对新的~MRPC~端盖~TOF~的离线数据刻度方法进行研究。

论文首先介绍了现在的~BESIII~探测器。在~BESIII~实验中参与粒子鉴别的是主漂移室和飞行时间探测器,主漂移室通过测量电离能损来提供~dE/dx~信息,飞行时间探测器测量带电粒子的飞行时间信息。粒子的鉴别能力由相同动量的不同种类的带电粒子的飞行时间差和飞行时间探测器自身的时间分辨决定。原来的端盖飞行时间探测采用的是单层的塑料闪烁体,对于~$\pi$~介子的时间分辨是~138~ps,已经达不到~BESIII~实验高精度测量的要求。MRPC~探测器具有时间分辨小,造价低廉等优点,可以用来作为新型的飞行时间探测器。2015~年夏季~BESIII~的端盖飞行时间探测器完成了硬件的升级改造,由原来的闪烁体换成了~MRPC~,并且参与~2015-2016~年度运行取数。研究新的~MRPC~端盖飞行时间探测器的刻度方法,开发相应的刻度软件,刻度消除带电粒子在读数条内的传播时间和过阈时间的晃动问题,得到好的时间分辨,对于粒子鉴别和物理分析具有重要意义。利用真实数据刻度样本,对信号在读数条的传播时间,以及过阈时间等对时间分辨影响最为敏感的问题开展了研究。对于在读数条的传播时间问题,相对简单,采用多项式拟合即可完成。而对于过阈时间的修正问题,相对复杂,这也是本文研究的重点和难点。

STAR~实验采用的是样条插值方法。做法是先对事例起始时间修正,之后先对过阈时间修正,然后对击中位置修正。其中对过阈时间和击中位置采用两次修正的迭代方法,最终扣除事例起始时间后的时间分辨是~75~ps。ALICE~的实验采用的是多项式拟合的方法,扣除事例起始时间后的时间分辨是80~ps。

在本文的第三章讨论了插值方法在~BESIII~实验~MRPC~离线数据刻度中的应用。经过对过阈时间和击中位置修正先后顺序的研究。发现在~BESIII~实验上采用先对击中位置进行修正,然后对过阈时间采用样条插值得到的时间分辨好。具体做法是先对击中位置进行修正,这样时间和过阈时间的依赖减弱,然后在此基础上对过阈时间插值拟合,得到模块编号是~55,读数条编号是~7~这一条单端最终的时间分辨是~64~ps。第四章介绍了公式在离线数据刻度中的应用,依然采用先对击中位置进行刻度,然后对过阈时间采用公式拟合,选用了几种公式进行拟合,并比较拟合修正后得到的时间分辨和时间对击中位置,过阈时间等的分布,其中~${p_{0}+p_{1}/\sqrt{q}+p_{2}/q}$~是重点考虑的公式,这时候这一条单端的时间分辨是~63.9~ps。在第五章中介绍了双端刻度的方法。由于~MRPC~端盖飞行时间探测器采用双端读出,对应的一个事例在双端各有一个原始时间信号和过阈时间信号。通过分析两端信号的均值,进行刻度修正,先对过阈时间进行修正,和单端的方法类似。修正过阈时间后,时间对击中位置还有一些依赖。考虑击中位置信息来自径迹外推,而径迹外推在读数条两端的边界处有很大误差。而两端时间差是和击中位置呈线性关系的量。因此采用两端时间差对击中位置进行修正,最终得到的双端的时间分辨是~${p_{0}+p_{1}/\sqrt{q}+p_{2}/q}$~公式是~53.5~ps,插值方法是~53.2~ps。

在论文的第六章对刻度公式的适用性问题进行了分析,MRPC~共有~72~块,每块有~12~个读数条,共有~864~个读数条,刻度公式需要具有一定的适用性。选用如下公式:

单端:$p_{0}+p_{1}/\sqrt{q}+p_{2}/q+p_{3}*zrhit+p_{4}*zrhit^{2}+p_{5}*zrhit^{3}+p_{6}*zrhit^{4}$

双端:$p_{0}+p_{1}/\sqrt{q}+p_{2}/q+p_{3}*t_{sub}+p_{4}*t_{sub}^{2}+p_{5}*t_{sub}^{3}+p_{6}*t_{sub}^{4}$\\
其中对于单端:~q~表示的是过阈时间(TOT),~zrhit~表示的是击中位置;对于双端:~q~表示的是双端的过阈时间的平均值,$t_{sub}$~表示的是测量的初始时间差的一半,即~$t_{sub}$=($t_{left}$-$t_{right}$)/2,与zrhit~量存在线性关系,可以用来修正击中位置。经过整体刻度,得到一组刻度常数,发现刻度常数是稳定的。查看了时间分辨随模块编号和读数条的编号基本是稳定的。整体~864~条读数条的单端时间分辨是~67~ps,双端是~56.6~ps。之后研究了过阈时间和击中位置关联项在刻度公式中的影响,发现关联很弱,几乎可以忽略。

论文最后对TOF的离线刻度软件进行了介绍。刻度软件基于BESIII实验的BOSS软件框架,利用面向对象的程序设计思想,主要采用C++语言开发完成。TOF的离线刻度算法基于最小二乘法原理,相应的刻度参数的求解可以转化为对矩阵方程:F$\times$X=Y中N$\times$N的矩阵F和N维向量Y的求解。在软件设计中充分利用C++语言的封装,继承和多态等性质。关于run号,事例数号,模块号,读数条编号,读出端编号等数据类型的定义,刻度和重建流程采用统一的数据定义格式。软件中求解矩阵F和向量Y的成员函数采用虚函数的定义方式,设计成为一个统一的接口,对于桶部和端盖,以及其他不同的刻度内容,可以很容易的重载出不同形式的算法,实现了一个接口下,对应不同刻度内容具有不同刻度算法的功能。软件兼具良好稳定性的同时,具有好的可扩展性,可以方便的利用继承机制添加其它的刻度内容项,而不会影响软件现有的功能。利用软件测试版本对离线数据进行刻度,刻度常数和时间分辨随读数条,模块数稳定性良好。刻度公式和刻度算法软件的健壮性需要在今后的工作中更加的完善。

论文研究得到了离线刻度算法,完成了相关的软件开发,将优秀的探测器设计指标转化为物理分析中优秀的粒子鉴别能力,将会对BESIII实验获得高精度的测量结果产生积极的促进作用。

\begin{comment}
论文利用~BESIII~获取的~Bhabha~事例真实数据刻度样本,对升级改造后的~MRPC~端盖~TOF~的离线数据刻度算法进行了研究,确定了刻度的流程,构造了合理的刻度公式,完成了相关的软件开发工作,刻度后得到的单条的时间分辨双端达到~53.5~ps,整体~72~个模块~864~条的双端时间分辨达到~56.6~ps,优于原来的端盖闪烁体的时间分辨(110~ps)。国际上另外使用~MRPC~作为飞行时间探测器的~STAR~实验和~ALICE~实验选用刻度样本是~$\pi$介子,时间分辨分别为~75~ps和~80~ps。

MRPC端盖TOF离线数据刻度算法是基于BOSS软件平台,利用面向对象思想开发的软件包,对原始数据中的噪声本底具有一定的排除能力。但是由于加速器和探测器的复杂性,原始数据中的本底情况可能具有很大差别,刻度算法软件的健壮性需要在日常刻度工作中逐步完善。本次研究得到的离线刻度算法,将优秀的探测器硬件指标转化为物理分析中优秀的粒子鉴别能力,将会对BESIII实验获得高精度的测量结果产生积极的促进作用。
\end{comment}












