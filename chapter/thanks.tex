% !TeX root = ../main.tex
% !TEX root = ../main.tex
% -*- root: ../main.tex -*-
% -*- program: pdflatex -*-
\begin{thanks}

三年光景,不过弹指一挥间。前些日子的一个下午在主楼看到那些在四楼大厅焦急等待硕士复试结果的一群学生。不禁回想起自己3年前来所里复试的情景。那时候自己不也和他们一样渴望进入高能所,对未来满怀憧憬。如今,自己也已经完成了自己硕士期间的课题。在此,要对那些曾经在成长路上帮助过我的那些人表示感谢。

首先,需要感谢的是我的导师孙胜森老师。记得和孙老师第一次见面时,他就说年轻人要目标长远,理想远大。孙老师学识渊博,对BESIII探测器软件部分很是了解,更是TOF探测器方面的专家。刚入所时,我对TOF探测器的硬件以及软件都不甚了解,程序也看不懂。在孙老师的帮助下和一次次耐心的讲解下,对BESIII实验的软件有了比较全面的了解,熟悉了TOF探测器的刻度和重建的流程,也能够自己编写一些程序脚本对数据进行处理和检查。在之后对MRPC数据离线刻度方法的研究中更是每次遇到困难都能得到孙老师好的指导和建议。而且孙老师指导学生认真负责,每此的考核报告,都从内容,逻辑,格式,用词等方面对我提出意见。除了在学术上对我的帮助外,孙老师还经常和我聊一些科研工作外的话题,教我一些科研外的东西。临近毕业,我有些迷茫,也对自己有些不满,孙老师也给予了我很大的鼓励和安慰。

然后需要感谢一下林韬师兄。林韬师兄在计算机方面知之甚多,而且很是热心肠。在B406办公室,我和他是邻桌,刚入室的时候,我对编程可谓一窍不通,这个时候林韬时候就经常主动帮助我,教我怎么写结构体和类,教我如何编译。让我慢慢的能够自己写代码,并对程序的运行等等都有了一些了解。他还经常推荐我一些网站和教我一些软件和工具,像github,有道云笔记,IHEPBox等都是他介绍给我并教我如何使用的。我的这些论文的latex模板也是他提供给我的。这里我之所以单独感谢一下林韬师兄是因为大多数情况下,都是他主动帮助我的。这是一个人最难能可贵的品质。

感谢戴忠、于庆洋、董伟伟、付颖、张其安、魏占辰、邴丰、杨佼汪、黎炎等等同学,在怀柔一起去上课,一起去图书馆自习,一起在操场打牌,一起吃串喝酒等等,总之在怀柔有你们这帮同学的陪伴,我有了一个丰富多彩的学习生活。

感谢教育处的保增宽老师,李苏敏老师,陈红珍老师,柯笑晗老师。来所里复试时接待的是你们,感谢你们在入学,考核,各类档案材料以及之后论文提交等手续给予的帮助。在我打印成绩单需要所里出具证明的时候,在我学生证需要注册的时候,在我对毕业论文的事情有些不清楚询问的时候,每位老师总是很热情。尤其需要感谢的是柯笑晗老师,记得中期考核报告第一次提交的那一份中有一个明显的打字错误,于是我把新的版本又一次邮件发给了柯老师,那时候已经是晚上8点50分了,而第二天的早上8:30分就要考核了,我很担心自己的报告来不及更新。而第二天报告的时候,我的报告已经是更新过的。我想,如果报告来不及更新,那个明显的打字错误有很大的可能会对我这次考核的得分有影响。总之,你们对自己的工作认真负责,对学生态度友善。

感谢B406办公室的安芬芬、李新颖、王洪鑫、张坤、周明、方肖、王蒙、肖言佳,马明明、杨荣兴、黄震、张晋、付婷婷、苗楠楠、陆佳达等师兄师姐师弟师妹在学习和生活上对我的帮助和照顾。

感谢父母、弟弟对我学业上的支持。我体质原因,容易上火,每次你们电话都是让我多注意休息,多吃水果。很感谢父母在外打拼,努力赚钱,让我不至于为了家里的生计担忧,能够静下心来好好学习,安心科研。感谢父母在我每次心情不爽的时候可以安慰我,鼓励我。感谢弟弟和他女朋友在16年元旦来看我,给我带礼物,并能够在节假日或者北京天气有变化的时候给我打电话,发短信。你们永远是我最坚强的后盾。


\end{thanks}













